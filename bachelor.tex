\documentclass[dvipdfmx]{jsreport}
\usepackage{amsmath,bm,ascmac,braket}
\begin{document}



\title{中性子検出器HIMEの性能評価と開発}
\author{池田旭輝}
\maketitle
\tableofcontents
\clearpage

\chapter{序}
加速器の技術発展に伴い安定核から遠く離れた不安定核の研究が進み,中性子の束縛境界である中性子ドリップライン付近の原子核では陽子を含むコア部分から遠く離れたところに中性子の波動関数がしみ出す中性子ハロー現象,例えば$^{31}\mathrm{N}$で現われた通常の核構造では魔法数であるはずの$N=20$が破れるという魔法数消失,逆に新たな魔法数が現われる新魔法数の出現などの特異な現象が現われることが分かってきた.そのような中性子過剰領域で予見される性質の一つとして,二中性子が空間的にコンパクトな領域で強く相関する''ダイニュートロン相関''がある.以下ではこのダイニュートロン相関に関する説明と本研究の関連について述べる.
自由空間において束縛する二核子は,陽子と中性子の重陽子のみであり,さらにスピンが反平行であるという制限が付く.二中性子は自由空間においては散乱長$a_{nn}=-18.2(6)$fmと非束縛の核である.しかし1970年代,ミグダルは原子核表面において強い相関をもち束縛する二中性子系が存在しうると理論的に予言した.この相関をダイニュートロン相関という.
この予言以降,ダイニュートロン相関の探索実験がいくつか行われ,存在の兆候が見られた実験もあったが,今なお直接の観測はされていない.
\\
ダイニュートロン相関が現われる可能性が示唆されているのは中性子過剰核や二中性子ハロー核の表面などがある.中性子ハロー核とは通常の原子が原子核の周囲に電子が存在するように,通常の密度のコア核の周囲を中性子が分布している原子核のことである.この中性子ハロー核の特徴として,通常の核であれば8MeV程度の分離エネルギーより遙かに低い1MeV以下の分離エネルギー,三番元素Liのハロー原子核が82番元素Pbの断面積に匹敵するほどの大きな断面積,後述するソフト$E_1$励起などがある.この二中性子ハロー核におけるダイニュートロン相関探索実験が行われている.
\\
ダイニュートロン探索実験の先例として,二中性子ハロー核である$^11$Liを用いた実験がある.
この実験は理化学研究所RIBFで行われた.この実験では,クーロン分解反応という手法を用いた.クーロン分解反応とは,鉛のような原子番号の大きい原子核の標的の近傍を高エネルギーで目的核を通過させ,それによって生じる電磁場の揺らぎによるパルスを仮想光子の吸収とみなして目的核を励起させ,分解させるという手法である.このクーロン分解反応の断面積から$E_1$遷移強度$B(E_1)$を求め,和則によって目的核である二中性子ハロー核の構造を求めることが可能となる.和則は以下のようにして計算される.\begin{align}B(E_1)&=\frac{3}{4\pi}\left(\frac{Ze}{A}\right)\braket{(\bm{r}_1+\bm{r}_2)^2}\\&=\frac{3}{4\pi}\left(\frac{Ze}{A}\right)\braket{r_1^2+r_2^1+2r_1r_2\cos{\theta}_{12}}\end{align}.ここで$\bm{r}_1,\bm{r}_2$は二つのハロー中性子のそれぞれの位置ベクトル,$\theta_{12}$はハロー中性子同士がなす角である.
ハロー中性子が核内で無相関であった場合は,二中性子の開き角の期待値は$\braket{\theta_{12}}=90度$となるのに対して,相関があり二中性子が空間的にコンパクトな領域に局在していた場合は,有意に90度より小さくなる.この実験によって得られた$E_1$遷移強度の分布が下図である.0.6MeV付近のピークは,ハローとコアの分極によって生じるソフト$E_1$励起である.この結果より,核内での二中性子の開き角として$\braket{\theta_{12}}=48^{+14}_{-18}$度と,無相関の場合である90度より有意に小さい開き角が得られた.これは核内で二ハロー中性子がダイニュートロン相関をもち,空間的にコンパクトな領域に局在していることの裏付けとなる.しかし,この実験はモデル依存性が高いという問題点が指摘されている.この実験においては$^{11}$Liのコア核である$^9$Liが基底状態であることを前提としているが,実際は励起状態の可能性があると指摘する研究もある.\\
$^{11}$Liのような束縛核だけでなく,非束縛核にもダイニュートロン相関は現われる可能性がある.
以下に,$^{13}$Liで行われた$^{13}\mathrm{Li}\rightarrow^{11}\mathrm{Li}+n+n$崩壊で放出された二中性子の開き角の分布を示す.
ここで,$\theta_k$はヤコビ座標系のT系,Y系の角度であり,ヤコビ座標系は以下のような図と式で直交座標系と変換される.
\begin{align}
\bm{k}_x&=\frac{m_1\bm{k}_1-m_2\bm{k}_2}{m_1+m_2}\\\bm{k}_y&=\frac{m_3(\bm{k}_1+\bm{k}_2)-(m_1+m_2)\bm{k}_3}{m_1+m_2+m_3}\\\cos{\theta}_k&=\frac{\bm{k}_x\cdot\bm{k}_y}{k_xk_y}\\E_x&=\frac{(m_1+m_2)k_x^2}{2m_1m_2}
\end{align}
この分布図より,二中性子はY系における180度に近い角度で放出されたことが分かる.すなわち,同方向に放出されやすかったことを示す.また,以下には$^{16}$Beの二中性子崩壊の角度$\theta_{nn}$の分布を示す.
この場合の角度はヤコビ座標系ではなく,下図のようなニ中性子間の角度で定義される.
こちらの場合も$\theta_{nn}$は0度付近が多く観測されているため,同方向に放出されやすかったことを示している.
ここで,核外で同方向に放出されているということは,二中性子の相対運動量が小さいことを表しており,不確定性原理により空間的には遠い,すなわちダイニュートロン相関とは逆の相関がこの核の中では生じていることが分かる.
対して$^{26}$O原子核では萩野,佐川は180度に近い方向,すなわち反対方向に放出される可能性が高いと予測した.
実際に$^{26}$Oの角度相関を調べる実験がMSU(Michigan State University)で行われた.下図はそのシミュレーションであり,赤がダイニュートロン相関の逆の相関があるとしたもの,青がダイニュートロン相関,緑は無相関のモデルである.この実験の結果が下図である.シミュレーションと比較しているものの,この実験においては検出器の分解能が足りず,ダイニュートロン相関か無相関か,ダイニュートロン相関の逆の相関かを識別するには至らなかった.よって,さらなる実験においては分解能の高い検出器が必要となる.
\\
将来におけるダイニュートロン探索実験の候補としてみられているのが$^{26}$O原子核である.この原子核は非束縛であるためクーロン分解反応のような手順を踏まずとも自発的に$^{26}\mathrm{O}\rightarrow n+n+^{24}\mathrm{O}$という二中性子崩壊を起こす.また,下図のようなエネルギーレベルのため,$^{26}\mathrm{O}\rightarrow n+^{25}\mathrm{O}$のような一中性子崩壊を起こさず$^{25}$Oの励起状態による影響を受けることがない.また,二中性子分離エネルギーは18keVとこれまでに観測されている中で最小であるため,二中性子間の相関が特に強く表れる.
ダイニュートロン相関は低密度核において現われる.ここでフェルミガスモデルによる密度はフェルミ波数$k_{F}$,フェルミ運動量$p_F$を用いて\begin{align}k_{F}=\frac{p_F}{\hbar}=(3\pi^2\rho)^{\frac{1}{3}}\end{align}と言う関係がある.二中性子分離エネルギー$S_{2n}$と運動量の間には$S_{2n}=p_F^2/2m_n$の関係があるため,密度の二中性子分離エネルギー依存性は\begin{align}\rho=\frac{1}{3\pi^2}\left(\frac{p_F}{\hbar}\right)^3=\frac{1}{3\pi^2}\frac{(2m_nS_{2n})^{\frac{3}{2}}}{\hbar^3}\end{align}
となる.よって密度と二中性子分離エネルギーの間には$\rho\propto S_{2n}^{\frac{1}{3}}$の関係があり,分離エネルギーの小さい原子核で密度が下がることが分かる.
よって最小の二中性子分離エネルギーをもつこの$^{26}$O原子核はダイニュートロン相関探索実験の対象として注目を集めている.
次に実験計画をを記す.核子あたり220MeVのエネルギーで運動する$^{27}$Fを液体水素標的で一陽子分離反応を起こして$^{26}$Oにし,自発的に崩壊した$^{24}$Oと二中性子を検出する.$^24$Oも中性子もエネルギーは250MeVとなっている.このとき$^{24}$OはSAMURAIの磁場によって曲げられ,60度軌道を変えて検出される.中性子は磁場によって軌道が変えられることはないため,直進してくる.
この時,二中性子の開き角は12mrad程度であり,標的から11m離れた地点においては二中性子間の距離は14cm程度となる,
従来の中性子検出器NEBULAでは,そのシンチレータの幅は14cmで分解能は数cm程度であるため,その精度は十分とは言い難い.
よってより薄く分解能がより中性子高い検出器が必要となる.
\\

\chapter{中性子検出}
\section{中性子検出の原理}
中性子は電荷を持たないため,陽子や電子のような電磁相互作用を用いた検出は不可能である.そのため,中性子と検出器の原子核との強い相互作用による反跳荷電粒子を検出することで間接的に中性子を検出することで運動量を測定する.この章では,中性子検出器HIMEに使われるプラスチックシンチレータを用いた中性子検出の原理について述べる.
プラスチックシンチレータが用いられる理由には,この材質が安価で加工しやすく,容易に大型の検出器を作ることが出来ること,Hを多量に含むため中性子との相互作用によって反跳した陽子のエネルギーが大きくなること,時間応答がよいことが挙げられる.
ダイニュートロン探索実験において中性子が持つエネルギーは数百MeV程度である.そのエネルギー領域でプラスチックシンチレータ内で中性子が起こす反応は主として以下の五つである.\begin{equation}\begin{split}&1.n+p\rightarrow n+p\\&2.n+^{12}\mathrm{C}\rightarrow n+^{12}\mathrm{C}
\\&3.n+^{12}\mathrm{C}\rightarrow \gamma+X
\\&4.n+^{12}\mathrm{C}\rightarrow p+X
\\&5.n+^{12}\mathrm{C}\rightarrow n+n+X
\end{split}\end{equation}
この中で中性子検出に用いられるのは,陽子を反跳する1と4の反応である.
2,3,5の反応は検出できず,バックグラウンドとなるためスレッショルドをかけて除去する.
以下にそれぞれの反応について詳細を書く.\\
弾性散乱において,入射粒子から反跳粒子へのエネルギー移行率は非相対論的なエネルギー保存と運動量保存を用いて\[\frac{E}{E_n}=4\cos{\theta}^2\frac{M}{(1+M)^2}\]となる.
ここで,$E,E_n$はそれぞれ反跳粒子と入射中性子のエネルギー,$\theta$は散乱角,$M$は反跳粒子の質量/中性子の質量である.このとき,0度散乱に対して陽子と中性子の間の散乱であれば,$E/E_n=1$となるのに対し,$^{12}\mathrm{C}$では$E/E_n=48/169\sim0.28$と小さくなる.
次に,軽い粒子と重い粒子のシンチレータ内での発光量の差について解説する.シンチレータ内での荷電粒子の発光量$dL/dx$とエネルギー損失$dE/dx$の関係式はBirksの式\[\frac{dL}{dx}=\frac{S\frac{dE}{dx}}{1+kB\frac{dE}{dx}}\]と表される.ここで,$S$はシンチレーション効率,$k$が消光の割合,$B$は定数である.電子の場合は,運動エネルギーが約1MeVとなったところで最小電離損失粒子(MIP:Minimum Ionizing Particle)となり,これ以上のエネルギーでは発光量とエネルギーの関係はBirksの式により線形で近似される.ここから発光量の単位として電子が1MeVのエネルギーを失った時の発光量を1MeVee(MeV electron equivalent)と定義される.
プラスチックシンチレータ内部での荷電粒子の入射エネルギーに対する発光量は下図のようになる.
この図より,電荷と質量の大きい粒子ほど発光量は相対的に小さくなることが分かる.
$^26$Oの実験において中性子は250MeVのエネルギーを持って入射するため,反跳する$^{12}\mathrm{C}$のエネルギーは70MeV程度である.70MeVの$^{12}\mathrm{C}$に対して陽子の場合は250MeVがそのまま入射する.
以上より,陽子の発光量に対して$^{12}\mathrm{C}$の発光量は非常に小さく,この検出器に対しては不感となる.
3の反応で放出される$\gamma$線は最大で4.44MeVeeで発光するため,スレッショルドをそれ以上にかけて排除することは可能である.
5の反応で放出される中性子は,後述するクロストークの原因ともなるがこの中性子が検出されるためには5の反応を起こした上でもう一度中性子による1か4の反応を引き起こす必要があり,その可能性は低いため除外される.
\section{中性子の運動量測定}
中性子の運動量ベクトルの測定は飛行時間法(TOF法:Time Over Flight)で行われる,運動量測定は粒子の位置と時間の測定が必要である.
中性子検出器の位置と時間を$(x,y,z,t)$とすると中性子の相対論的な運動量ベクトルは
\begin{equation}\begin{split}&\beta=\frac{\sqrt{x^2+y^2+z^2}}{ct}\\&\gamma=\frac{1}{\sqrt{1-\beta^2}}\\&p=m\gamma\beta\\&\vec{p}=p\frac{\vec{r}}{\sqrt{x^2+y^2+z^2}}\end{split}\end{equation}
となる.$c$は光速,$m$は中性子の質量である.よって,中性子の運動量ベクトルを求める為に中性子の位置と時間の精密測定が必要である.

\section{クロストーク}
クロストークとは,一つの中性子の入射に対して複数モジュールがシグナルを出す現象のことである.複数モジュールが反応する事象には一つの中性子が複数のモジュールで反応しシグナルを出す事象,反跳した陽子が複数のモジュールでシグナルを出す事象,クロストークでなく複数の中性子がそれぞれ一モジュールで反応を起こす事象がある.
中性子の測定において,クロストークと複数中性子の事象を区別することが必要である.クロストークによって生じた二度目以降の反応は入射した中性子の運動量とは異なるため,除去する必要がある.しかし,シグナルが一中性子によるものか二中性子によるものかを区別することは複数中性子のイベントでは区別できない.\\

理研のNEBULAなどの同じくプラスチックシンチレータで構成される複数層を持つ検出器では,シグナル間の速度差を用いて複数中性子イベントかクロストークかを判別している.下図がその選別の模式図である.クロストークである場合は,一度目の反応で中性子は必ず速度を落とすため,一度目の反応が起きたモジュールでの速度$\beta_{01}$と二度目の反応が起きたモジュールでの速度$\beta_{12}$の関係は$\beta_{01}>\beta_{12}$となる.対して複数中性子イベントの場合,二つのイベントに関係はないため,$\beta_{01}$と$\beta_{01}$の間に相関はない.よって$\beta_{01}<\beta_{12}$のイベントのみを抜き出すことで複数中性子イベントを選別することが可能となる.しかし$\beta_{01}>\beta_{12}$の複数中性子イベントは失われてしまう.
一方HIMEの場合,下図に示すようにモジュールの厚さが既存の中性子検出器よりも薄い.そのため反跳陽子を複数のモジュールで検出することが可能である.複数モジュールで検出された陽子の反応場所の記録から飛跡を再構築し,クロストークと複数中性子イベントを区別するという手法が考案されている.

\chapter{中性子検出器HIME}
\section{旧来のHIMEの構造}HIMEは,プラスチックシンチレータで構成される中性子検出器である.
中性子検出に用いるNEUTは
この1モジュールを10本並べたものを1層とし,この層を縦横直交に4層並べたものとモジュール8本を層としたもの1層の計5層がHIMEの中性子検出部(NEUT)である.
実際に中性子検出を行う各層が重なっている部分の面積は40cm$\times$40cmである.
荷電粒子が入射してきた場合でもNEUTはシグナルを検出するため,中性子のシグナルと荷電粒子のシグナルを区別するべくNEUTの前には幅360mm,高さ1050mm,厚さ10mmのプラスチックシンチレータであるVETOを配置する.
中性子の反応率は1$\%$/cm程度であるのに対して,荷電粒子の反応率はほぼ100$\%$であることから,VETOとNEUTがともにシグナルを出しているというイベントを除外することで,中性子によるイベントのみを取り出すことが可能となる.
先述した様にプラスチックシンチレータは中性子が反跳させた荷電粒子を検出する.
プラスチックシンチレータの蛍光を両端に接続されたPMTによって増幅させ,電気信号に変換させてデータを得る.
HIMEはこの5層のNEUTとVETOという構成であったが,ドイツのミュンヘン大学の協賛によってこの構成をもう一層追加する運びとなった.よってHIME本体がより磁場に近づくことになり,それによる影響がどれほどかを調べることが必要となった.よって本実験では,磁場の影響を調べるために横置きのプラスチックシンチレータ24本での実験的な検出器を作り,ここに2.7Tと言う磁場をかけた時のPMTへのゲインの影響を調べた.
\section{HIMEのアップグレードの計画}
\section{テスト実験器}

\section{HIMEの信号処理回路とデータ収集}
シンチレータから送られたパルスはToTディスクリミネーターと増幅器の役割を持つPaDiWaにまず送信される.PaDiWaは16チャンネルまでを検出可能である.PaDiWaに送られた信号はTDCに送られる.このHIMEの場合は,到着した信号に対してゲートをかけるためのFPGA(field programmabl gate arrays)五つがTRB3(TDC readout boards)に装着されているという外観となっている.
この中央のFPGAは他の4つのFPGA(周辺FPGAと呼ぶ)の信号を調整するCTS(central trigger system)としての役割を担っている.また,CTSはDAQとの通信も行っている.
周辺FPGAはそれぞれに三つまでPaDiWaからのケーブルを接続することが可能である.
一つ一つのPMTにかけた電圧は1300V
次に,モジュ-ルの情報処理の方法について述べる.
モジュールのシンチレータ内で反応が起こった後,プラスチックシンチレータ内で発生した光は左右のPMTでそれぞれ観測される.
左のPMTが光を観測した時間を$T_l$,右が観測した時間を$T_r$とすると,時間差tDiffと時間の和tSumが\begin{align}\mathrm{tDiff}&=T_r-T_l,\\\mathrm{tSum}&=T_r+T_l\end{align}のように記録される.このとき,PMTの左右とはSAMURAIのビームラインの上流から見ての左右である.
HIMEを稼働させて宇宙線との反応によって得られた生のデータをルートファイルにして記録されるのはtDiff,tSum,左のtot,右のtot,モジュールのID,その反応は何層突き抜けたものかを示すnHitsである.
tot(time over threshold)は下図に示すように反応が起きてイベントの波形が立ち上がり,スレッショルドを超えた時間と減衰してスレッショルドを下回った時間の差によって得られる.
このFPGAの中で時間を計る仕組みについて解説する.
このfineカウンターはその日の温度によって変化するため,キャリブレーションのデータはその日その日に行うことが必要となる.下に具体的な温度との関係を示す.
最大で16度の温度差に対して120psという差が現われる,
モジュールの数は24本であり,合計48チャンネルがFPGAに接続されている.以下にそのチャンネルとPMTのモジュールの対応関係を示す.

\chapter{HIMEの性能評価}
新たに24本のモジュールを並べた実験用のHIMEの性能確認と磁場がどのような影響を及ぼすかを調べるため,理研SAMURAIエリアで二つの実験を行った.

\section{宇宙線を用いた時間分解能の算出}
HIMEの位置分解能がダイニュートロン探索実験に必要な水準を満たしているかどうかを確認するため,宇宙線を用いてHIMEの時間分解能を評価した.
時間分解能$\sigma_t$と位置分解能$\sigma_x$の間の関係は比例関係\begin{equation}\sigma_x=v\sigma_t\end{equation}であるため,時間分解能から位置分解能は求められる.
まずは,得られたデータからの位置のキャリブレーションを行う.
先述したとおり,このTDCの
\section{実験のセットアップ}
宇宙線による時間分解能の評価を行う際の
\section{磁場環境下におけるHIMEのゲインへの影響}
磁場環境下におけるPMTへの影響として,PMT内部での電子が磁場によるローレンツ力によって軌道を曲げられ,電子増倍の過程や最終的な電子の収集が現象することである.
HIMEの改装が終わった後予定されている実験には$^{8}\mathrm{He}(p,p\alpha)4n$,$^{6}\mathrm{He}(p,3p)4n$によるテトラニュートロンの観測実験,$^{8}\mathrm{He}(p,3p)6n$によるヘキサニュートロンの観測実験がある.この実験において予想されている磁場の磁束密度の値は$^{6}\rm{He}$において2.0T,$^{8}\rm{He}$の実験において2.7Tである.
よって,磁場による影響を調べるために,本研究ではAmBe線源による2.7Tの磁場下のHIMEのゲインと非磁場下のゲインを比較した.
\subsection{実験セットアップ}
本実験でのHIMEの配置は前節の時間分解能の評価の際と同じである.SAMURAIの磁石は回転台の端から5m離れており,よってHIMEから磁石の中心までの距離はとなる.
\subsection{磁場による影響}
下に示す図が磁場をかけていないときとかけているときのAmBe線源によるヒストグラムの比較である.
この二つのヒストグラムを重ね書きしたヒストグラムが以下となる.
この図より,磁場の有無によるPMTへのゲインはないということがわかる.
\chapter{まとめと展望}
\section{実験のまとめ,考察}
\section{今後の展望}
\end{document}